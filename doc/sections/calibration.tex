\chapter{Camera calibration}\label{ch:calibration}

In order to carry out the project, the robot arm was acompained by a camera that provided the necessary information to determine the position of the bricks. However, the use of this camera required a calibration process which was done by means of a checkerboard. This step was essential in order to find the intrinsic and extrinsic parameters of the camera, which were main tool to perform transformations between frames. Additionally, these parameters were also used to undistort the images taken by the camera and reduce the "fish-eye" effect that the camera might include.

The calibration of the camera was performed using the \textit{Camera Calibration Toolbox} for MATLAB. The process was based on thirty pictures from different angles of the checkerboard. In each image, four different corners were selected (the first corner was the origin) as well as defining the size of various squares. This process was repeated in the same order for every picture and all this data collected in a matlab file. The first step of this calibration computes a closed-form solution for the calibration parameters without considering the lens distortion. This step is followed by a nonlinear optimization where the main goal is to minimize the total reprojection error over all the calibration parameters, both intrinsic and extrinsic. This procedure is done iteratively based on the \textit{gradient descent method} and, finally, the result is a set of variables describing the camera characteristics. 

In addition, the calibration results also show the errors associated with some variables allowing us to evaluate the quality of the calibration. Moreover, based on the plot of the errors provided by the calibration, some pictures were removed from the calibration pictures pool (the ones with the biggest error values). After eliminating the worst pictures, the image corners were recomputed automatically using the function \textit{Recomp. Corners} and then the \textit{Calibration} function was applied once again in order to get new parameters with an even smaller error.

As mentioned above, the camera calibration was also used to undistort the background image and the pictures of the workspace. This process was essential to remove the deformation in the external areas and, thus, obtain the best position values of the blocks. \par


\section{Intrinsic and Extrinsic Parameters}
The parameters provided by the calibration of the camera can be used to make transformations between frames which is useful to calculate the correct position of the bricks. These parameters can be divided in two different groups: the intrinsic and the extrinsic. The intrinsic parameters are related to the camera (equation \ref{intrinsic}), for instances, the position of the principal point, the skew parameter (is zero because the axis are orthogonal) and the focal length. Using these parameters it is possible to combine the coordinates of the camera frame with the ones in the picture (pixels). The extrinsic parameters are the ones responsible for the relation between the world coordinates and the camera coordinates as is possible to see in \ref{extrinsic}.

\begin{align} 
\label{intrinsic}
\begin{bmatrix}
    \textit{u} \\ 
    \textit{v} \\
    \textit{w} 
\end{bmatrix}
=
\begin{bmatrix}
    \textit{f}  & 0 & p_{x} & 0\\
    0   &  \textit{f} & p_{y} & 0  \\
    0 & 0 & 1 & 0 
\end{bmatrix}
\begin{bmatrix}
   x_{s}\\
   y_{s}\\
   z_{s}\\
	1
\end{bmatrix}
\text{  , K}
=
\begin{bmatrix}
    \textit{f}  & 0 & p_{x} & 0\\
    0   &  \textit{f} & p_{y} & 0  \\
    0 & 0 & 1 & 0 
\end{bmatrix}
\end{align}

\begin{align} 
\label{extrinsic}
\begin{bmatrix}
   x_{s}\\
   y_{s}\\
   z_{s}\\
	1
\end{bmatrix}
=
\begin{bmatrix}
    \textit{R}  & \textit{T}
\end{bmatrix}
\begin{bmatrix}
   X_{w}\\
   Y_{w}\\
   Z_{w}\\
	1
\end{bmatrix}
\end{align}

Based on the relations between frames, it is possible to simplify the calculations associating the image coordinates with the world coordinates. The relation is made through a \textit{Projective matrix} ($3\times 4$ matrix) which is a result of the product between the intrinsic ($3\times 3$ matrix) and extrinsic ($3\times 4$ matrix) matrices.

\begin{align} 
\begin{bmatrix}
    \textit{u} \\ 
    \textit{v} \\
    \textit{w} 
\end{bmatrix}
=
\textit{K}
\begin{bmatrix}
    \textit{R}  & \textit{T}
\end{bmatrix}
\begin{bmatrix}
   X_{w}\\
   Y_{w}\\
   Z_{w}\\
	1
\end{bmatrix}
=
\textit{P}
\begin{bmatrix}
   X_{w}\\
   Y_{w}\\
   Z_{w}\\
	1
\end{bmatrix}
\end{align}

The \textit{Projective matrix} is a $3\times 4$ matrix which means that it is not invertible. However, it is possible to assume that \textit{Z} is always zero because the blocks are in the \textit{XY} plane and thus the third column of the \textit{projective matrix} will be multiplied by zero. So, it is possible to ignore the third column of the \textit{projective matrix} and the third row of the matrix with the world coordinates.

\begin{align} 
\begin{bmatrix}
    \textit{u} \\ 
    \textit{v} \\
    \textit{w} 
\end{bmatrix}
=
\textit{P}
\begin{bmatrix}
   X_{w}\\
   Y_{w}\\
	1
\end{bmatrix}
\end{align}

As a result, \textit{P} can be represented as $3\times 3$ matrix which means that it is invertible, as it is possible to observe in \ref{invertible}. 

\begin{align}
\begin{bmatrix}
   X_{w}\\
   Y_{w}\\
	1
\end{bmatrix} 
=
\textit{P}^{-1}
\begin{bmatrix}
    \textit{u} \\ 
    \textit{v} \\
    \textit{w} 
\end{bmatrix}
\label{invertible}
\end{align}

