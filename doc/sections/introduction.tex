\chapter{Introduction}\label{ch:introduction}

Since the Industrial Revolution, the industry has been evolving, increasing the number of machines in factories in order to satisfy the production demands and to reduce the effort made by the workers. However, in the last few years, the industry motto has been changing, influenced by the development of the technology. The main goal is to decrease the number of people working in firms, reducing, thus, the cost with manpower, and increasing the number of autonomous machines. These mechanisms are able to work without breaks for long periods with a high level of accuracy. 

For instance, nowadays robot arms are one of the most used tools in factories around the world because they are able to perform precise tasks without making any mistakes. This instrument is a type of mechanical arm, divided in diferent parts, with similar functions to a human one. The desired rotation and translation motion are achieved based on joints which are the ones responsible for the connection between the links. 

With this background in mind, the objective of this project is to build figures which represent the Simpson characters. In order to reach the goal it was necessary to implement a system with two main items: one camera and one robot (ADEPT Cobra). The camera was responsible for taking pictures which were submited to a computation process based on colour and edge detection. So as to decrease the execution time, the background was subtracted from the images with bricks and then, the outcome was cropped. Hence, it was possible to perform a quick analysis of all images and, consequently, it optimized the entire program.

After transforming the coordinates and after using the information provided by the camera, it was possible to move the robot and to pick the blocks. This movement is based on the inverse kinematics due to the fact that the robot knows the final position of the gripper but it needs to calculate the best combination of angles to achieve the desired location.
