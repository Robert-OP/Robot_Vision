\chapter{Results, Discussion and Conclusions}\label{ch:conclusion}

Two different approaches for building the figures have been addressed on the project. They will be refered through the text as \textit{"On the fly"} and \textit{"Regular"} construction.

The \textit{"Regular"} construction consists of the steps described in "ref to chap 2" and it is shown in the first video. 

The "On the fly" construction differs a bit in some of the steps with the "Regular" one.
For starters, this second form of building will only take \textbf{one} picture per figure. This results in a faster construction since taking a new picture and making the whole process to calculate new positions of the blocks every time takes quite a bit of time.
However, this sort of process introduced a new problem to the table, since we are choosing the block that is going to be picked based on the maximum area and, in some cases, the same color will have to be picked twice.
For this reason a "color counter" has been added during this method, keeping track of how many times a certain color has been picked and therefore calculating the position of the desired block as the relative maximum area with respect to the counter. I.e.: If one color has to be picked more than once, the block with the second biggest area would be picked the second time, the one with the third biggest area would be picked the third time, and so on. 

On the other hand, even though the "On the fly" construction is significantly faster, it is also way less \textbf{robust}. This is due to the fact of taking only one picture per each figure construction. 
We can conclude that a compromise between speed and robustness should be achieved in a real scenario, since new blocks can be added to the workspace while a figure is being constructed or even considering the fact that the first (and only) picture is taken in the process might happen to be of not very good quality (blurred).

\paragraph{} ASDFASDF
